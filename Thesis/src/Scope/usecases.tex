\section{Use Cases}

	As the target audience consists of software developers with little or no prior experience with graphics engines, the library will provide the most convenient API for the simplest tasks. We considered the following list to be the most rudimentary tasks when confronted with a graphics engine API for the first time:

	\begin{smalllist}
		\item Creating an empty scene and looking into that scene through a camera: This step involves the creation of all necessary objects to enter the rendering loop. As this step is part of every application, the reduction of this boilerplate code could be considered to be the most important use case.
		\item Positioning an object in the scene: Loading an object into the scene and altering its coordinates, orientation and/or scale to be visible through the default camera.
		\item Updating an object: Moving, rotating or otherwise modifying a previously loaded object discretely in a pre-defined time period.
		\item Controlling the camera: Updating properties of the camera, enabling the implementation of a dynamic scene.
	\end{smalllist}

	This list of use cases will be used during the evaluation of the final architecture in Section \ref{chapter:evaluation:tests}.

