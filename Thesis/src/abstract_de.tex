\begin{abstract}

	Beim Entwickeln von Grafik-Engines liegt des Augenmerk meist auf der Performance, gemessen an der Anzahl Frames pro Sekunde, die die Engine zu generieren vermag. Dieser starke Fokus auf die Effizienz führt in vielen Fällen dazu, dass sich auch performancerelevante Details der Implementierung in der API wiederfinden um eine möglichst effiziente Verwendeung des Produkts zu ermöglichen. Durch diese Herangehensweise wird für die Verwendung solcher APIs einiges an Wissen in Teilbereichen der Grafikprogrammierung vorausgesetzt.

	Wir wollen im Rahmen dieser Arbeit die Notwendigkeit dieser Herangehensweise überprüfen. Wir wollen herausfinden, ob eine Reduktion des benötigten Fachwissens einen direkten Performance-Verlust zur Folge hat. Wir haben zu diesem Zweck drei quelloffene Grafik-Engines analysiert, um Konzepte zu identifizieren, die außerhalb der Grafikdomäne wenig Verbreitung finden. Auf Basis unserer Erkenntnisse haben wir PURGE -- eine eigene Grafik-API -- entwickelt, die auf die Verwendung durch Entwickler ausgerichtet ist, die wenig bis überhaupt keine Erfahrung in dieser Domäne besitzen.

	Anschließende Tests haben ergeben, dass diese zusätzliche Software-Ebene kaum Auswirkungen auf die Performance von visuell einfachen Applikationen hat.

\end{abstract}

