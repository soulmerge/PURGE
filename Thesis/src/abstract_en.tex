\begin{abstract}

	When referring to the quality of 3D graphics engines, the central point of interest is usually computational performance. This very intense occupation with this aspect of such graphics engines leads to the exposure of performance-relevant implementation specifics in the API, to allow performance-aware usage of the product, but limiting the audience to those knowledgeable about the details of graphics engine implementation.

	Our aim was to evaluate whether the amount of domain knowledge in existing graphics engine APIs was justified; whether it would be possible to reduce the knowledge expected from the user of such APIs without sacrificing too much performance. We have analyzed three different open source graphics engines to identify elements that are less known outside of the graphics development domain. Based on our findings, we have created PURGE, our own high-level API for graphics development, which makes use of other libraries for the actual rendering.

	We have found that the performance impact of the added software layer was not noticeable during the development of simple test scenes, which we assume to be the prior concern of developers who are starting graphics development for the first time.

\end{abstract}

