\section{Motivation}

    Familiarizing oneself with a new domain requires one to learn and understand new concepts. In the case of 3D graphics engines\footnote{The term \emph{graphics engine} is used for any software library providing an API for graphical computation with a visual output.}, this means learning how a graphics engine works on the inside. The level of abstraction provided by many such library APIs is enough for developers with existing knowledge of this domain to adapt to a new API, but we found many of the libraries to provide an insufficient API for those without prior exposure to graphics development.

    Graphical computation is a vast domain, where library-developers focus on the performance to achieve better-looking and/or more complex scenes. This very intense occupation with this aspect of such graphics engines leads to the exposure of performance-relevant implementation specifics in the API, to allow performance-aware usage of the final product, but limiting the audience to those knowledgeable about the details of graphic engine implementation.

	Since advances in graphics hardware have made it possible for a modern graphics engine to supply several orders of magnitude more performance over the last decades, another approach to designing APIs for such engines will be explored in this document - one that focuses on ease of use, possibly sacrificing performance for this goal.

